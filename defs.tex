% Ordinary (value) types (ty-)
\newcommand{\ty}[0]{\tau} % Meta variable for types
\newcommand{\tyHref}[1]{\keyword{Href}\left<#1\right>} % Hole reference type, parameterized by the type within the hole
\newcommand{\tyNum}[0]{\keyword{Num}}
\newcommand{\tySum}[2]{#1 + #2}
\newcommand{\tyProd}[2]{#1 \times #2}

% Expression forms (e-)
\newcommand{\e}[0]{e} % Meta variable for expressions
\newcommand{\ePalLet}[3]{\keyword{let palette}\,#1\,\keyword{=}\,#2\,\keyword{in}\,#3}
\newcommand{\ePalApp}[2]{#1~#2}

% Palette types (pt-)
\newcommand{\pt}[0]{T} % Meta variable for palette types
\newcommand{\ptPal}[2]{\keyword{Palette}\,#1\,#2}
\newcommand{\ptArr}[2]{#1 \rightarrow #2}

% Palette expressions (p-)               % 
\newcommand{\p}{p} % Meta variable for   % 
                   % palette expressions % From Ravi's notes:
\newcommand{\pDef}[1]{#1}                %   pd
\newcommand{\pLam}[2]{\lambda#1.#2}      %   \x -> p
\newcommand{\pApp}[2]{#1~#2}             %   p e

% Palette definitions (pd-)
\newcommand{\pd}{D} % Meta variable for palette definitions
\newcommand{\pdBody}[7]{\{ #1, #2, #3, #4, #5, #6, #7 \}} % (the body of) a palette definition
\newcommand{\pdBodyTwoRows}[7]{\left\{ \begin{array}{l} #1, #2, #3, \\ #4, #5, #6, #7 \end{array} \right\}} % (the body of) a palette definition

% Palette typing context (ascribes a palette type to each palette variable)
\newcommand{\Ptc}[0]{\Delta}        % Meta variable
\newcommand{\PtcEmp}[0]{\cdot}      % Empty
\newcommand{\PtcBind}[3]{#1, #2:#3} % Binding

% Value typing context (ascribes a value type to each value variable)
\newcommand{\Ctx}{\Gamma}           % Meta variable
\newcommand{\CtxEmp}[0]{\cdot}      % Empty
\newcommand{\CtxBind}[3]{#1, #2:#3} % Binding
